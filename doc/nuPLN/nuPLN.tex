%% Towards a complete formalization of PLN
%%
%% Render with: ./render.sh

\documentclass[]{article}
\usepackage{url}
\usepackage{minted}
\usepackage[hyperindex,breaklinks]{hyperref}
\usepackage{breakurl}
\usepackage{listings}
\lstset{basicstyle=\ttfamily\footnotesize,breaklines=false,frame=single}
\usepackage{float}
\restylefloat{table}
\usepackage{longtable}
\usepackage{graphicx}
\usepackage[font=small,labelfont=bf]{caption}
\usepackage[skip=0pt]{subcaption}
\usepackage{circledsteps}
\usepackage{amsfonts}

\begin{document}

\newcommand{\nuPLN}{\nu\textup{PLN}}
%% \newcommand{\Bool}{\mathbb{B}}
\newcommand{\Bool}{\textup{Bool}}
%% \newcommand{\Bool}{B}
\newcommand{\T}{\top}
\newcommand{\F}{\bot}
\newcommand{\Domain}{\mathcal{D}}
\newcommand{\Language}{\mathcal{L}}
\newcommand{\Event}{\mathcal{F}}
\newcommand{\Model}{M}
\newcommand{\Data}{D_{x\in\Domain}}

\title{Towards a Complete Formalization of PLN\\
  \texttt{(DRAFT)}} \author{Nil Geisweiller}
\maketitle

\section{Introduction}
The goal is similar to Solomonoff Universal Induction~\cite{TODO},
that is we want to approach a first order (unknown but computable)
distribution $\mu$ given observations, using a second order
(uncomputable but known) distribution $\nu$, called the Universal
Distribution.  In Solomonoff Induction, observations are bit strings
produced by a Turing machine\footnote{Note that even though the sample
space of $\nu$ is made of deterministic Turing machines, $\nu$ can
approximate any computable distribution $\mu$ (thus non-deterministic)
by maintaining an ensemble of such Turing machines.}.  In PLN however,
observations are outcomes from an indexed boolean random variable,
representing the outputs of evaluating a predicate on some inputs.
Such predicate is called the \emph{observable predicate}.  In practice
PLN allows multiple observable predicates however one can assume one
predicate without loss of generality.  Indeed, to emulate multiple
predicates, one can introduce an extra component in the predicate's
domain to ``select'' the predicate of interest.  Also, since it is
observed by a random variable, such predicate is not necessarily
deterministic (though it could be).  As such, one may think of the
observable predicate as being a program drawn from a certain
Probabilistic Programming Language.  In the following section we
formally define the above.

Because this reformulation of PLN departs somewhat from the definition
of PLN in the PLN book~\cite{TODO}, we give it a new name, $\nuPLN$.

\section{Definitions}
Given an observable predicate $\mu$ with domain $\Domain$, let
$(\Omega, \Event, \nu)$ be a probability space such that
\begin{itemize}
\item $\Event$ is the event space, a $\sigma$-algebra on $\Omega$.
\item $\nu : \Event \to [0, 1]$ is a universal distribution, further
  defined below.
\item $\Omega$ is the sample space associated to $\nu$, containing
  mappings of inputs to outputs, that is records of evaluations of
  $\mu$ from $\Domain$ to its Boolean co-domain, as well as all
  potential definitions of $\mu$\footnote{Those familiar with
  Solomonoff Universal Induction, please note that here $\mu$
  represents a probabilistic predicate rather than a computable
  probability function calculating the probability of any event,
  although the latter can be derived from the former.}.
\end{itemize}
Since $\mu$ is a probabilistic predicate, its type signature cannot
merely be
$$\mu : \Domain \to \Bool$$ where $\Bool = \{\F, \T\}$.  To capture
its probabilistic nature we give it the following type signature
$$\mu : \Domain \to \Omega \to \Bool$$ in a curried fashion.  Meaning
that given its argument, it produces a boolean random variable.
Therefore the observable predicate can be viewed as an indexed boolean
random variable.  Also, in cases where the domain $\Domain$ can be
decomposed into multiple components, a curried notation will be
preferred.  So instead of
$$\mu : (\Domain_1 \times \dots \times \Domain_n) \to \Omega \to \Bool$$
the following
$$\mu : \Domain_1 \to \dots \to \Domain_n \to \Omega \to \Bool$$ will
be used.  This will be convenient to express inheritance relationships
between partially applied predicates.  Applying $\mu$ to an input $x$
will be denoted with the following functional programming notation
$$(\mu\ x)$$ Likewise, $\mu$ applied to more than two arguments will
be denoted
$$(\mu\ x\ \omega)$$ and so on.  For $\mu$ we use such functional
programming notation is used because it is currying-friendly.  For the
rest we use the traditional function application notation, such as
$$\nu(E)$$ In case $\mu$ is known to be deterministic, $\Omega$ could
potentially be dropped, but that is not going to be our working
assumption for now.  With that, let us now define key random variables
to access $\Omega$:
\begin{itemize}
\item $\Model : \Omega \to \Language$ with measurable space
  $(\Language, \Event_\Language)$, where $\Language$ is a certain
  probabilistic programming language and $\Event_\Language$ is a
  $\sigma$-algebra on $\Language$.  Thus, $\Model$ takes a particular
  world $\omega \in \Omega$ and outputs the probabilistic program $\mu
  \in \Language$ generating that world.  Note that this random
  variable is inaccessible from an observer within that world.  An
  observer within that world only has access to a finite record of
  evaluations of $\mu$.  However, that random variable will be
  convenient to define aspects of the semantics of $\nuPLN$.
\item $\Data : \Omega \to \Bool$, a family of boolean random variables
  indexed by values in $\Domain$, the predicate inputs.  Unlike
  $\Model$, $\Data$ is at least partially accessible from an observer
  within that world.  Meaning, such observer can gather data for a
  finite subset of $\Domain$.  $\Model$ and $\Data$ are related by the
  following equality
  $$(\mu\ x\ \omega) = (D_x\ \omega)$$ where $\omega$ runs over all
  elements of $\Omega$ such that $(M\ \omega) = \mu$.
  %% Thus,
  %% equivalently and more formally
  %% $$\forall \omega \in \Omega, \mu \in \Language, x\in \Domain,
  %% (M\ \omega) = \mu \Rightarrow (\mu\ x\ \omega) =
  %% (\Domain_x\ \omega)$$
  %% In other words, $D$ provides a window into the behavior of $\mu$.
\end{itemize}
Due to the equality above, the distribution of observations is
entirely determined by a model.  In other words, it suffices to define
a distribution over $\Language$, the prior distribution, to define
$\nu$, as far as $M$ and $\Data$ are concerned anyway.  Then, relating
observations to models can be done using regular Bayesian inference.
The prior is thus defined by
$$\nu(M \in L)$$


Let us denote the probability space
of this prior as

$$(\Language, \Event_\Language, \nu_\Language)$$

\begin{itemize}
\item $\nu_\Language$ is a distribution over $\Language$.
\item $\Event_\Language$ is the event space associated to
  $\nu_\Language$.
\item $\Language$ is the sample space of $\nu_\Language$.
\end{itemize}
If you wonder why I am being so pedantic, carefully defining the
probability space over $\Language$, it is because $\Language$ is not
going to be your average discreet Turing machine language.  For the
purpose of recovering PLN with this Bayesian approach, $\Language$
will have to be continuous.  I will explain in detail why, but before
that let us provide the definition of $\nu$ in terms of its prior
$\nu_\Language$

$$NEXT$$
%% $$\nu : \Event \to [0, 1]$$

\section{conclusion}

\bibliographystyle{splncs04}
\bibliography{local}

\end{document}
